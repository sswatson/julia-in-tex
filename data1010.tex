% !TEX program = lualatex
% !TEX options = --shell-escape

\documentclass[prettycode,jlmin,shellescape]{watsonbook}

\HideEnvironment{solution}

\begin{document} 

\pagecolor{white} 

\newpage

\chapter*{Preface} \thispagestyle{empty}

In this book, we develop the mathematical ideas underlying the modern
practice of data science. The goal is to do so accessibly and with a
minimum of prerequisites.* \sidenote{* Experience with basic calculus
  is necessary, and some prior experience with linear algebra and
  programming will be helpful.} For example, we will start by
reviewing sets and functions from a data-oriented perspective. On the
other hand, we will take a problem-solving approach to the material
and will not shy away from challenging concepts. To get the most out
of the course, you should prepare to invest a substantial amount of
time on the exercises.

This text was originally written to accompany the master's course DATA
1010 at Brown University. The content of this PDF, along with hints
and solutions for the exercises in this book, will be available on the
edX platform starting in the fall semester of 2019, thanks to the
BrownX project and the Brown University School of Professional
Studies.

The author would like to acknowledge Isaac Solomon, Elvis Nunez, and
Thabo Samakhoana for their contributions during the development of the
DATA 1010 curriculum. They wrote some of the exercises and many of the
solutions that appear in the BrownX course.

\newpage

\tableofcontents

\newpage

\chapter{Programming in Julia} \label{ch:programming-in-julia} 

In this course, we will develop mathematical ideas in concert with corresponding computational skills. This relationship is symbiotic: writing programs is an important ingredient for applying mathematical ideas to real-world problems, but it also helps us explore and visualize math ideas in ways that go beyond what we could achieve with pen, paper, and imagination.

We will use \textit{Julia}. This is a relatively new entrant to the
scientific computing scene, having been introduced publicly in 2012
and reaching its first stable release in August of 2018. Julia is
ideally suited to the purposes of this course:
\begin{enumerate}
\item \textbf{Julia is designed for scientific computing}. The way
  that code is written in Julia is influenced heavily by its primary
  intended application as a scientific computing environment. This
  means that our code will be succinct and will often look very
  similar to the corresponding math notation. 
\item \textbf{Julia has benefits as an instructional language}. Julia
  provides tools for inspecting how numbers and other data structures
  are stored internally, and it also makes it easy to see how the
  built-in functions work. 
\item \textbf{Julia is simple yet fast}. Hand-coded algorithms are
  generally much faster in Julia than in other user-friendly languages
  like Python or R. This is not always important in practice, because
  you can usually use fast code written by other people for the most
  performance-sensitive parts of your program. But when you're
  learning fundamental ideas, it's very helpful to be able to write
  out simple algorithms by hand and examine their behavior on large or
  small inputs. 
\item \textbf{Julia is integrated in the broader ecosystem}. Julia has
  excellent tools for interfacing with other languages like C, C++,
  Python, and R, so can take advantage of the mountain of scientific
  computing resources developed over the last several
  decades. (Conversely, if you're working in a Python or R environment
  in the future, you can write some Julia code and integrate it into
  your Python or R program).
\end{enumerate}

\section{Environment and workflow}

There are four ways of interacting with Julia: 
  \begin{enumerate}
  \item \textit{REPL}. Launch a read-eval-print loop from the
    command line. Any code you enter will be executed immediately,
    and any values returned by your code will be displayed. 
  \item \textit{Script}. Save your code in a file called
    \texttt{example.jl}, and run \jlverb{julia example.jl} the
    command line to execute all the code in the file.
  \item \textit{Notebook}. Like a REPL, but allows inserting text and
    math expressions for explanation, grouping code into blocks,
    multimedia output, and other features. The main notebook app is
    called \textbf{Jupyter}, and it supports Julia, Python, R and many
    other languages.
  \item \textit{IDE}. An integrated development environment is a
    program for editing your scripts which provides various
    code-aware conveniences (like autocompletion, highlighting, and
    many others). \textbf{Juno} is an IDE for Julia. 
  \end{enumerate}

  Some important tips for getting help as you learn:

  \begin{enumerate}
  \item Julia's official documentation is available at
    \url{https://docs.julialang.org} and is excellent. The learning
    experience you will get in this chapter is intended to get you up
    and running with the ideas we will use in this course, but if you
    do want to learn the language more extensively, I recommend
    investigating the resources linked on the official Julia learning
    page: \url{https://julialang.org/learning/}
  \item You can get help within a Julia session by typing a question
    mark before the name of a function whose documentation you want to
    see. 
  \item Similarly, \jlverb{apropos("eigenvalue")} returns a list
    of functions whose documentation mentions "eigenvalue"
  \end{enumerate}

  \begin{exercise}{}{julia-install}
   Install Julia 1.0 on your system, start a command-line session
    by running \jlverb{julia} from a terminal or command prompt,
    What does the ASCII-art banner look like (the one that pops up
    when you start a command line session)? 
  \end{exercise}

  \begin{solution}
    It looks like this: 
    \begin{textblock}[title=]
                   _
       _       _ _(_)_     |  Documentation: https://docs.julialang.org
      (_)     | (_) (_)    |
       _ _   _| |_  __ _   |  Type "?" for help, "]?" for Pkg help.
      | | | | | | |/ _` |  |
      | | |_| | | | (_| |  |  Version 1.0.3 (2018-12-18)
     _/ |\__'_|_|_|\__'_|  |  Official https://julialang.org/ release
    |__/                   |
    \end{textblock}
\end{solution}
  
  \section{Fundamentals}

  We begin by developing some basic vocabulary for the elements of a
  program.
  
\subsection{Variables and values}

A \textbf{value} is a fundamental entity that may be manipulated by a
program. Values have \textbf{types}; for example, \jlverb{5} is
an \jlverb{Int} and \jlverb{"Hello world!"} is a
\jlverb{String}. Types are important for the computer to keep track
of, since values are stored differently depending on their type. You
can check the type of a value using the \jlverb{typeof} function:
\jlverb{typeof("hello")} returns \jlverb{String}. 

A \textbf{variable} is a name used to refer to a value. We can
\textbf{assign} a value (say \jlverb{41}) to a variable (say
\jlverb{age}) as follows:
\begin{juliablock}[title=]
  age = 41 
\end{juliablock}

A \textbf{function} performs a particular task. For example,
\jlverb{print(x)} writes a string representation of the value
of the variable \jlverb{x} to the screen.  Prompting a
function to perform its task is referred to as \textbf{calling}
the function. Functions are called using parentheses following the
function's name, and values needed by the function are supplied
between these parentheses.

An \textbf{operator} is a special kind of function that can be
called in a special way. For example, the multiplication operator
\jlverb{*} is called using the mathematically familiar
\textit{infix notation} \jlverb{3 * 5}, or
\jlverb{*(3,5)}.

A \textbf{statement} is an instruction to be executed. For
example, the assignment \jlverb{age = 41} is a statement.  An
\textbf{expression} is a combination of values, variables,
operators, and function calls that a language interprets and
\textbf{evaluates} to a value. For example,
\jlverb{1 + age + abs(3*-4)} is an expression which
evaluates* \sidenote{\texttt{abs} is the absolute value function}
to \jlverb{54}.

A sequence of statements or expressions can be collected into a
\textbf{block}, delimited by \jlverb{begin} and
\jlverb{end}. The statements and expressions in a block are
executed sequentially. If the block concludes with an expression,
then the block returns the value of that expression. For example,
\begin{juliablock}[title=]
  begin
      y = x+1
      y^2
  end
\end{juliablock}
is equivalent to the expression \jlverb{(x+1)^2}.

The \textbf{keywords} of a language are the words that have
special meaning and cannot be used for other purposes. For
example, \jlverb{begin} and \jlverb{end} are keywords in
Julia.

\begin{exercise}{}{substitution-block}
  What value does the following expression evaluate to? What is the
  type of the resulting value? Explain. 
  \begin{juliablock}[title=]
    begin
        x = 14
        x = x / 2
        y = x + 1
        y = y + x
        2y + 1
    end
  \end{juliablock}
  \hint{Walk through the substitutions one at a time, starting from
    the top.}
\end{exercise}

\begin{solution}
  This expression returns the \jlverb{Float64} value
  \jlverb{31.0}. In the second line, the value associated with
  the variable \jlverb{x} is changed to \jlverb{7.0},
  because dividing integers returns a float in Julia. In the third
  line, the value 8 is stored in the variable \jlverb{y}, and in
  the next-to-last line, that value is replaced with
  \jlverb{15.0}. Finally, the last line evaluates to
  \jlverb{31.0}, and the whole block evaluates to the value of
  the final expression. 
\end{solution}

\subsection{Basic data types}

Julia, like most programming environments, has built-in types for
handling common data like numbers and text.

\begin{enumerate}
\item Numbers.
  \begin{enumerate}
  \item A numerical value can be either an \textbf{integer} or a
    \textbf{float}. The most commonly used integer and floating point
    types are \jlverb{Int64} and \jlverb{Float64}, so named
    because they are stored using 64 bits. We can represent integers
    exactly, while storing a real number as a float often requires
    slightly rounding it (see the \textit{Numerical
      Computation} chapter for details). A number is stored as a float or
    integer according to whether it contains a decimal point, so if
    you want the value 6 to be stored as a float, you should write it
    as \jlverb{6.0}.
  \item Basic arithmetic follows order of operations: 
    \jlverb{1 + 3*5*(-1 + 2)} evaluates to \jlverb{16}. 
  \item The exponentiation operator is \jlverb{^}. 
  \item Values can be compared using the operators* \sidenote{* For
      the last two symbols, use \texttt{\textbackslash le«tab»} and
      \texttt{\textbackslash ge«tab»}}[-10mm]
    \jlverb{==,>,<,≤,≥}.
  \item Julia integers will overflow unless you make them big${}^\dagger$
    \sidenote{${}^\dagger$ Operations with big-type numbers are much slower
      than operations with usual ints and floats}, so
    \jlverb{big(2)^100} works, but \jlverb{2^100}
    evaluates to \jlverb{0}. 
  \end{enumerate}
\item Strings
  \begin{enumerate}
  \item Textual data is represented in a \textbf{string}. We can
    create a string value by enclosing the desired string in quotation
    marks: \jlverb{a = "this is a string"}. 
  \item We can find the number of characters in a string with the
    \jlverb{length} function: \jlverb{length("hello")}
    returns \jlverb{5}. 
  \item We can combine two strings with the asterisk operator (\jlverb{*}):
    \jlverb{"Hello " * "World"}
  \item We can insert the value of a variable into a string using
    \textit{string interpolation}:
    \begin{juliablock}[title=]
      x = 4.2
      print("The value of x is $x")
    \end{juliablock}
    % $ <- DELETE THIS
  \end{enumerate}
\item Booleans
  \begin{enumerate}
  \item A \textbf{boolean} is a special data type which is either 
    \jlverb{true} or \jlverb{false}
  \item The basic operators that can be used to combine boolean values
    are  
    \begin{center} 
      \renewcommand{\arraystretch}{1.2}
      \begin{tabular}{c|c}
        and & \jlverb{&&} \\
        or & \jlverb{||} \\
        not & \jlverb{!} 
      \end{tabular}
    \end{center}
  \end{enumerate}
\end{enumerate}

\begin{exercise}{}{quick-brown-fox}
  \begin{enumerate}[(i)]
  \item Write a Julia expression which computes
    $\frac{1}{a+\frac{2}{3}}$ where $a$ is equal to the number of
    characters in the string
    \jlverb{"The quick brown fox jumped over the lazy dog"}
  \item Does Julia convert types when doing equality comparison? In
    other words, does \jlverb{1 == 1.0} return \jlverb{true}
    or \jlverb{false}? 
  \end{enumerate}
  \hint{(i) Use the \jlverb{length} function, (ii) try it out.}
\end{exercise}

\begin{solution}
  \begin{enumerate}[(i)]
  \item We store the length of the given string in a variable
    \jlverb{a} and evaluate the given expression as follows: 
    \begin{juliablock}[title=]
      a = length("The quick brown fox jumped over the lazy dog")
      1/(a+2/3)
    \end{juliablock}
  \item Yes, Julia does convert types for equality comparison. So
    \jlverb{1 == 1.0} returns \jlverb{true}. 
  \end{enumerate}
\end{solution}

Here's an example showing how to incorporate Julia output into your document: 

\begin{juliablockc}[title=][]
   a = 2
\end{juliablockc}
 
The value of \jlverb{a^200} is \jl{a^200}, but the value of \jlverb{big(a)^200} is 

$$\jl{big(a)^200}.$$

\end{document}